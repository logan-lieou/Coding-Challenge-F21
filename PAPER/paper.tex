\documentclass[12pt]{article}
\title{Fall 2021 ACM Coding Challenge}
\usepackage{cite}
\author{Logan Jackson}
\date{\today}
\begin{document}
\maketitle{}

\newpage
\section{Original Approach}
In the Python solution the approach was originally going to be just to take the entire input.txt file and feed it into a pre-trained transformer. But unfortunately that didn't work. So on the second try I decided to split up the contents of input.txt and pass parts of the file to the model one at a time. This is problematic due to the way Transformers work, Transformers are highly reliant on context, and due to this process of splitting up the file I am denying that context to the transformer thus having to process each string individually giving less accurate results.

The next problem we face with this approach is how to interpret the outputs in a way that is valid, the output of distilbert is the probability that the input matches a label this is basically the model’s confidence in its prediction i.e. an output of 0.99 is very confident and an output of 0.5 is the model being unsure. Attached to the confidence is a label, or the prediction of of whether the sentiment is positive or negative this is why the lower bound is 0.5 because after a probability of 0.5 it will just swap to predicting the other option. Throughout the input we see that the model fluctuates between positive and negative predictions. The way I get the overall sentiment of the input is by adding all positive predictions to all negative predictions, where positive predictions are positive and negative predictions are negative, the result will give me which predictions the model has more of.

\section{Numbers}
Using the approach I previously discussed we obtain the number 0.022 meaning that the model is tending slightly towards it’s positive predictions. Since each prediction will be positive or negative the sum could be very positive or very negative, actually there is no bound to how positive or negative a result can be therefore we need to squish the result, I do this using tanh if we have a input that is super, super positive say every sentence is 0.99 positive and there’s 100 sentences that means score is 99 by squishing this value with a tanh function we are able to find that this same value is just extremely close to 1.0 this process of squishing is called normalization and is commonly used to make unbounded outputs make sense. Passing the sum of output probabilities, to the tanh function we get 0.022 this normalized score is the exact same as our previous score, well it’s actually really close it’s not exactly 0.022 it’s more like 0.02199 but the score is very close to 0.022 so for convenience and rounding I will say that our normalized score is 0.022 based on this we can conclude that the sentiment of the input is slightly more positive than negative.

\newpage
\section{VADER Approach and Numbers}
The pretrained model thinks \{'neg': 0.065, 'neu': 0.748, 'pos': 0.187, 'compound': 0.9982\}
this means 6.5\% of the text has a negitive valence
score 74\% has a neutral valence score and 18\% has a positive
valence score.
the model seems to belive that the text is mostly neutral
now the issue here lies in "compound score"
on the vaderSentiment github page it says:
\begin{verbatim}
positive sentiment: compound score >= 0.05
neutral sentiment: (0.05 < compound score > -0.05) 
negative sentiment: compound score <= -0.05
\end{verbatim}
This is not repersentitive of what compound score actually is while looking
at the source code ~\cite{vadercode}
you can find how the compound score is actually calculated, based on our
output above it implies that the input is very positive. But based on 
the way that compound score is actually calculated. We see that this is
not the case.

$$
\frac{sum_s}{\sqrt{sum_s+\alpha}}, sum_s = \Sigma{}p_i n_i
$$

the problem lies in the way that VADER works and how it handles neutral
sentiment. Due to the way valence score works VADER will have the tendency
to produce true neutral results, a word will have +4, then another word -4 ~\cite{vadercode}
this is also another issue with VADER it's based on individal words predetermined
sentiment score meaning that VADER doesn't adapt sentiment scores based on context
this is problematic when dealing with larger inputs where sentiment may be highly varied
and context reliant. So back to the issue of our compound score.
Since the score is 0.99 you would expect the overall sentiment of the paper to be
extremely positive but when looking at the break down of the percent of the paper
that is positive negitive and neutral, we can see that the paper is mostly
labeled neutral and having more positive than negitive so if anything the overall
sentiment of the paper should be slightly positive but not extremely positive.

So you may be wondering how do I know that a score of 0.99 is extremely positive?
this is due to the normalization function presented above what the normalization
does is effectively "squish" the score to be between -1 and 1, where -1 is very 
negative and 1 being very positive. In theory this works by having each paragraph
be labeled either positive or negitive and if you take the sum of all positives and
negitives then squish it to a scale based on the hyper parameter $\alpha$ you
should get the overall sentiment of the paper. The problem lies in the way that it
handles neutral scores, because it's a sum neutral scores are entirely ignored
so even if a paper is mostly neutral say 99\% neutral and 1\% positive the compound
score will come out to be 0.99 an extremely positive score. 

So what does this number actually repersent
if it doesn't repersent the overall sentiment of the paper due to the fact
that it doesn't take into account
neutral scores, and how do we fix this? The first question of what it repersents
is pretty easy it repersents the normalized sum of positive and negitive sentiments 
across an input, once again this is invalid because it doesn't take into account
neutral seniments by virtue of it being a sum. Second how do we fix this? Well
you can actually weight the neutral scores by averaging the valence scores rather
than adding them together.

$$
score_{avg} = \frac{sentiment}{numInputs}, 
compoundScore\frac{score_{avg}}{\sqrt{score_{avg} + \alpha}}
$$

The reason we normalize this sum is because if a paper is extremely positive 
or extremely negitive we will get an average score of $>1$ or $<-1$, thus why we
have to normalize the score. This is the main reason why the nltk solution
contains a fork of the vader source code because I had to modify the model in order
to get a number that was repersentitive of the overall sentiment of the input.
After all of this the sentiment score is $0.025$ or within $\pm0.05$ of $0.0$ giving
us a slightly positive but overall nuetral score.

\newpage
\section{Transformer from Scratch Approach}
Yeah so the model took to long to train.

\newpage
\section{Naive Bayes Approach}
This is the solution that I was able to do in julia as implementing a transformer then training it was taking too long I decided
to use Naive Bayes instead, this is built into a convient library in julia called TextAnalysis. Naive Bayes uses statistical infrence 
in order to determine the probability of an event given another event. There's an entire paper on why Naive Bayes seems to work
so well for sentiment analysis but I'm not going to go too much into it here basiclly know that Naive Bayes is really 
really jank. Luckily our input seemed to have worked fairly well with the model I trained, although there are known issues with
this model.
\subsection{Data and Cleaning}
For the training I'm using the SST2 dataset ~\cite{} % TODO
I produce a csv file for training with a python script called data.py that just appends the words to the sentiment scores.
After that I read in the csv file to julia and do some magic.
\begin{verbatim}
function preprocess_sst2(df)
   df = df[:, ["body_text", "sentiment values"]]
	rename!(df, ["text", "sentiment"])
   sentiments = map((x) -> x >= 0.5 ? "negative" : "positive",
                    df.sentiment)
   return df.text, sentiments, unique(sentiments)
end
\end{verbatim}
I used explore.jl to figure out what I needed to do to clean the data then used this function to get the things I needed for training
the Naive Bayes model. There's another function in the file for cleaning a twitter dataset but the model was overwhelmingly negitive
so I decided to use SST2 instead. What's going on here is that I select the columns I want, the text and sentiment columns, then I
rename the columns for easy access. Next I map over the sentiment values and label each as either positive or negitive, we'll come
back to this later. Lastly the function returns the text, sentiments (array of labels), and unique(sentiments). We can think of
unique(sentiments) as being the A and B in, $P(A)$ and $P(B)$.

\subsection{Training the Model}
For training the model we have a simple function.
\begin{verbatim}
model = let
   nbc = NaiveBayesClassifier(uniques)
      for (text, label) in zip(texts, labels)
         sd = prepare_string_doc(text)
         fit!(nbc, text, label)
      end
   nbc
end
\end{verbatim}
All the function is doing is taking in the possible events (A, B as mentioned before) then iterating over all the words and
all the labels in our dataset.

\subsection{Numbers}
The output of the model is $0.995$ what does this mean? This means the model belives there is a $99.5\%$ probability the input is
positive. How I got here here is that I summed all of the probabilities of positive and negitive events similar to how VADER works except
instead of a valence score it's a probability in a particular classification. So you may be wondering how I delt with the nuetrality problem
what if a score is nuetral say 0.5 which would be true nuetral for this model well the answer is jank you just ignore it, instead of saying
0.5 was true nuetral it's actually positive, and less than 0.5 is negitive this allows us to sum positive and negitive probabilities
this model also relies on the idea that in any given input space there is $<\alpha$ true nuetral sentiments, which is often true $\alpha$ in
this case is repersentitive of arbitrary statistical confidence in a result. After we sum the probabilities we get some crazy unbounded value
that is repersentitive of the probability that the sentiment is either positive or negitive in an unbounded distrobution. We normalize the value
using tanh -1 being most negitive and +1 being most positive. Our output is then the probability of either
positive or negitive in a normalized space. * I HAVE NO IDEA IF THIS IS TRUE
$$
score = tanh(\Sigma{}A_i B_i)
$$

\subsection{Naive Bayes vs VADER}
You may be thinking what makes our score diffrent from valence score in VADER and why it's okay for this model to have
an output like $0.995$ positive but it's not okay for VADER to have that same output. Do the problems with VADER simply
not apply to this model? So yes the problems with VADER do not apply here the reason is our model is outputting a probability
where as VADER's valence score is outputting the strength of a particular sentiment. Let's say a word is +4 another word is -3
then the valence score is 1 a max valence score is $\pm15$ this score normalized is 0.25 what this means is that the sentiment 
is decently positive. But with our model we take those two words as inputs and output the probability that it's positive and the
probability that it's negitive in this case we may get something like "positive" $\rightarrow$ 0.99, "negitive" $\rightarrow$ 0.01, meaning that our
model predicts that the sentence has a 99\% chance of being positive.

\section{Conclusion}
The best solution was probably the transformer, but unfortuntely I had no time. Based on the other results of Naive Bayes, VADER, and dilbert
I can conclude that input.txt file is mostly nuetral with a tendency towards positive sentiment. After reading the input myself I agree with
the models the begining of the file is nuetral/positive then there's one part that could be negitive near the begining middle then after that
the rest seems to be nuetral/positive.

\newpage
\bibliography{biblio.bib}{}
\bibliographystyle{plain}

\end{document}